\chapter[Не только положительные ряды]{Ряды с членами разных знаков или с членами --- комплексными числами.}

\begin{thm}
\textbf{(Признак Лейбница)} Пусть $\ssum a_k$ --- ряд со знакочередующимся членами,
которыепо модулю монотонно стремятся к нулю.
Тогда ряд сходится и остаток $|r_n| \ge |a_{n+1}| \ge |a_n|$.
\end{thm}
\begin{proof}
Пусть $a_1 > 0$. Тогда $\s_{2n} = \ssumn (a_{2k-1}+a_{2k})$ -- неубывающая последовательность, $\s_{2n+1} = a_1 + \ssumn(a_{2k} + a_{2k+1})$ --- невозрастающая последовательность,
$\s_{2n} \le S_{2n+1} = a_1+\ssumn( a_{2k} + a_{2k+1}) \le 0$
--- невозрастающая последовательность, $\s_{2n} \le \s_{2n+1}$.

Значит, $\s_{2n}$ --- неубывающая, ограниченная сверху последовательность.
$\s_{2n+1}$ --- невозрастающая, ограниченная снизу последовательность.
Они сходятся и, так как $\s_{2n+1}-\s_{2n}=a_{2n+1}= \overline{o}(1)$,
то имеют бощий предел $\s$.
\\
$$r_{2n+1} = \sum_{k=2n+2}^{\infty} a_k =
\underbrace{a_{2n+2}}_{\le 0} +  \sum_{k=n+2}^{\infty}\underbrace{\big(a_{2k-1} + a_{2k}\big)}_{\ge 0} =  \sum_{k=n+1}^{\infty} \underbrace{\big(a_{2k}+a_{2k+1} \big)}_{\le 0}$$
Следовательно,
$$a_{2n+2} \le r_{2n+1} \le0, |r_{2n+1}| \le |a_{2n+2}| \le |a_{2n+2}|$$
$$r_{2n} = \sum_{k=2n+1}^{\infty} a_k = \sum_{k=n}^{\infty} \underbrace{\big( a_{2k+1} + a_{2k+2} \big) }_{\ge 0} =
\underbrace{a_{2n+1}}_{\ge 0} + \sum_{k=n+1}^{\infty} \underbrace{ \big(a_{2k-1} + a_{2k}\big)}_{\le 0}$$
Следовательно $0 \le r_{2n} \le a_{2n+1} \le |a_{2n}|$
\end{proof}
\newpage
\subsection*{Преобразование Абеля}

$$\sum_{k=m}^n u_k v_k = \sum_{s=m-1}^{n-1} U_k \big( v_k - v_{k+1} \big) + U_nv_n-U_{m-1}v_{m-1} =$$
$$=\sum_{k=m}^n U_k(v_k-v_{k+1}) +  U_nv_n - U_{m-1}u_m,$$
$$ \text{где } U_n=\ssumn u_k, U_0 = 0, v_0 =0$$
\begin{proof}
\begin{multline*}
\sum_{k=m}^n u_k v_k = \sum_{k=m}^n \big(U_k-U_{k-1} \big) v_k = \sum_{k=m}^n U_kv_k - \sum_{k=m}^nU_{k-1}v_k=\\
=\sum_{k=m}^{n}U_k v_k - \sum_{k=m-1}^{n-1} U_kv_{k+1} = \sum_{k=m-1}^{n-1}U_k(v_k-v_{k+1})+V_nv_n-U_{m-1}v_{m-1}=\\
=\sum_{k=m}^{n-1} U_k (v_k - v_{k+1}) + U_nv_n - U_{m-1}v_m
\end{multline*}
\end{proof}
\subsection*{Последовательность ограниченной вариации.}
\begin{deff}
Последовательность $\{v_k\}_{k=1}^{\infty}$ называется последовательностью ограниченной вариации, если сходится ряд модулей разниц между соседними членами
\end{deff}
\begin{thm}
Последовательность действительных чисел  является последовательностью ограниенной вариации тогда и только тогда, когда её можно представить как разность двух неубывающих(невозрастающих) сходящихся последовательностей
\end{thm}
\begin{proof}
(Достаточность)
Монотонная сходящаяся последовательность $v_k$ является последовательностью ограниченнной вариации.
Действительно, если $v_k$ --- невозрастающая последовательность, то $\ssumn |v_k-v_{k+1}| = \ssumn(v_k-v_k+1)= v_1-v_{n+1}$
--- имеет предел при $n \to  \infty$, следовательно, последовательность имеет ограниченную вариацию.

Сумма, разность последовательностей ограниченной вариации являются последовательностями ограниченной вариации. Для любого числа $\alpha$ и VB-последовательности $v_k$ \; $\alpha v_k$ ограниченна.

(Необходимость) Последовательность $S_n = \sum_{k=1}^{n-1} |v_k-v_{k+1}|$ --- неубывающая, $G_n = \sum_{k=1}^{n-1} \big |v_k -v_k+1|+(v_k-v_{k+1})$ --- неубывающая последовательность, $v_n = v_1 - \sum _{k=1}^{n-1}(v_k - v_{k+1})= S_n + v_1 - G_n=S_n -(-v_1 + G_n)$
Домножением на минус единицу можно получить рвзность двух невозрастающей последовательностей.
\end{proof}
\begin{thm}
Если $v_n$ --- последовательность ограниченной вариации, то она сходится
\end{thm}
\begin{proof}
$ v_n = v_1 + \sum_{k=1}^{n-1}(v_{k+1} - v_k)$, ряд $\sum_{k=1}^{n-1}( v_{k+1} - v_k) $ сходится абсолютно.
\end{proof}
