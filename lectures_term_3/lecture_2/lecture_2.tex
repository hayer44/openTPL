\chapter{}
\section*{Признаки сходимости рядов неотрицательных числел}
\begin{enumerate}
\item
Если дан ряд $\ssum a_k$, то ряд сходится тогда и только тогда, когда последовательность частичных сумм $\s_n$ ограничена.
\item
(Признак сравнения) Если даны  ряды $\ssum a_k $ и $\ssum b_k , 0 \le a_k \le b_k, k>\mathbf K$, то из сходимости второго ряда следует сходимость первого, а из расходимости первого расходимость второго.
\begin{proof}
$$ \ssumn a_k \le \sum_{k=1}^{K} + \sum_{k=K+1} a_k \le \sum_{k=1}^{K} + \sum_{k=K+1} b_k \le  \sum_{k=1}^{K} + \sum_{k=1} b_k \le$$
Из сходимости второго ряда следует ограниченность частиных сумм первого, а значит и сходимость первого.\\
Если $n>K, p \ge 0$, то $\sum_{k=n+1}^{n+p}a_k \le \sum_{k=n+1}^{n+p} b_k$, поэтому из выполнения критерия Коши для второго ряда следует выполнение критерия Коши для превого. 
\end{proof}
\item
(Сравннения) Пусть $\ssum a_k $ и $\ssum b_k$ -- числовые ряды с неотрицательными членами. Если $$0 < \alpha \le \frac{a_k}{b_k} \le \beta < \infty, \forall k>K,$$
то числовые ряды одновременно сходятся или одновременно расходятся.
\begin{proof}
$0 \le a_k \le \beta b_k$, поэтому, если ряд $\ssum b_k$ сходится, то сходится $\ssum a_k$.\\
Так как $0 \le b_k \le \frac{a_k}{\alpha}$,
То если ряд $\ssum a_k$ сходится, то сходится и второй.
\end{proof}
\item
(Сравнения) Пусть $\ssum a_k $ и $\ssum b_k$ -- числовые ряды со строго положительными членами. Если
$$ \frac{a_{k+1}}{a_k}\le \frac{b_{k+1}}{b_k}, \forall k \ge K,$$
то из сходимости ряда $\ssum b_k$ следует сходимость ряда $\ssum a_k$
\begin{proof}
$$ \prod_{k+K}^{n-1}\frac{a_{k+1}}{a_k} \le \prod_{k+K}^{n-1}\frac{b_{k+1}}{b_k}\text{, т.е. } \frac{a_{n}}{a_K} \le \frac{b_{n}}{b_K}, n>K$$
Значит, $a_n \le \frac{a_K}{b_K}b_n, n>K$, поэтому из сходимости ряда $\ssum b_k$ следует сходимость ряда $\ssum a_k$
\end{proof}
TODO: Вставить замечание
\item
(ДАлабера) Пусть $\ssum a_k$ -- ряд с неотрицательными членами. Если
$$\frac{a_{k+1}}{a_k} \le q < 1 \text{при} n\ge K$$
то ряд сходится. Если
$$ \frac{a_{k+!}}{a_k} \text{при} n\ge K$$
То члены ряда не стремятся к нулю, и ряд расходится
\begin{proof}
Возьмём $b_k = q^k$ -- геометрическую прогрессию, ряд $\ssum q^k$ сходится.
Далее используем признак сравнение. Другой случай очевиден.
\end{proof}
\item
(Коши) Пусть $\ssum a_k$ -- ряд с неотрицательными членами. Если $ \sqrt[n]{a_k}\le q < 1$ при $k \ge K $,
то ряд сходится, а если $ \sqrt[k]{a_k}\ge 1$ для бесконечного числа номеров, то члены ряда не стремятся к нулю и ряд расходится.
\begin{proof}
Если  $\sqrt[n]{a_k}\le q < 1$, то повторому признаку ряд расходится. 
\end{proof}
\item
(Интегральный Маклорена-Коши) Пусть $f(x)$ -- неотрицательная невозрастающая функция на $ [1,+\infty] $. Тогда
$$ 0 \ssumn f(k) - \int\limits_1^{n+1}f(x)dx\le f(1)$$
Ряд и интеграл одновременно сходятся или одновременно расходятся.
\begin{proof}
$$ 0\le \ssumn \left( f(k) - \int\limits_k^{k+1}f(x)dx \right) = \ssumn f(k) - \int\limits_1^{n+1}f(x)dx = $$
$$ = f(1)+\sum_{k=2}^n \left( f(k) - \int\limits_{k-1}^k f(x)dx \right) - \int\limits_n^{n+1} f(x)dx \le f(1)$$
Тогда частичные интеграллы и суммы ограниченны одновременно.
\end{proof}
\end{enumerate}