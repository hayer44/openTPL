\chapter{}
\begin{deff}
Пусть $ \{a_k \}_{k=n}^{\infty} $ --- последовательность, занумерованная целыми числами начиная с $ n $ и далее по возростанию.\\
Тогда выражение вида $a_m+a_{m+1}+a_{m+2}+ \cdots = \sum_{k=m}^{\infty} a_k$ называется бесконечным рядом.\\
\end{deff}
Изменением нумерации общий случай можно свести к случаю $n=1$ или $n=0$.
Тоже можно получить при $n>1$ добавлением нулевых членов или заменой начальных членов их суммой в случае $n<0$.
\begin{deff}
$ \s_{N} = \sum_{k=m}^{N} $ --- частичная сумма с номером N.\\
$ \s_{N} = 0 $, если $ N < m $.\\
Если существует предел:
$$ \lim_{N \to \infty} {\s_N} = \s $$
то его называют суммой ряда $\s$. Если $a_k$ --- действительные числа, то $\s$ действительное число или $\pm \infty$.
Ряд называется сходящимся, если его сумма конечна. 
Если это не так, то ряд называют расходящимся.
\end{deff}
\begin{st}Критерий Коши.\\
Числовой ряд сходится тогда и только тогда, когда:
$$ \forall{\epsilon\!>\! 0} \quad \exists N \quad \forall m \! >\! N \quad \forall p \in \nat : \left| \s_{m+p} - \s_m \right| < \epsilon$$ 
\end{st}
\begin{st}Необходимое условие сходимости.\\
$$ \mbox{Если ряд } \sum_{k=1}^{\infty} a_k \mbox{ сходится, то } a_k \xrightarrow{k\to\infty}0$$
\end{st}
\begin{deff}
Если ряд сходится и $\s$ --- его сумма, то $r_n = \s - \s_n$ называется остатком ряда с номером n.%
\end{deff}
\begin{deff}
Ряд $\ssum a_k $ называют абсолютно сходящимся, если сходится ряд $\ssum |a_k| $.
\end{deff}
\begin{thm}
Если ряд $\ssum a_k$ абсолюно сходится, то он сходится.
\end{thm}
\begin{proof}
По критерию Коши, $ \forall{\epsilon\!>\! 0} \quad \exists N \quad \forall m \! >\! N \quad \forall p \in \nat : \left| \s_{m+p} - \s_m \right| < \epsilon$
Так как $\left| \s_{m+p} - \s_m \right| \le \sum_{k=m+1}^{m+p}|a_k|$, то выполняется критерий Коши для исходного ряда, и он сходится.
\end{proof}
\begin{deff}
Если ряд сходится, но не сходится абсолютно, то его называют сходящимся условно.
\end{deff}
\section*{Свойства}
\begin{enumerate}
\item
Если ряд $\ssum$ сходится (сходится абсолютно) и $\s$ --- его сумма, то для любого числа $\alpha$ ряд $\ssum \alpha a_k$ сходится.
\begin{proof}
$$\ssumn \alpha a_k = \alpha \ssumn a_k = \alpha \s_n$$
Если сущестует предел частичных сумм исходного ряда равный $\s$, то $\s_\alpha = \alpha \s$
\end{proof}
\item
Если ряды $\ssum a_k$ и $\ssum b_k$ сходятся (абсолютно сходятся)
и $\s_a$ и $\s_b$ --- их суммы, то ряд $\ssum(a_k+b_k)$ сходится (сходится абсолютно) и его сумма --- $\s_a+\s_b$
\begin{proof}
$$\ssumn(a_k+b_k) = \ssumn a_k + \ssum b_k \xrightarrow{n\to\infty} \s_a + \s_b$$
Так как $$ \ssumn|a_k+b_k|\le \ssumn|a_k| + \ssumn|b_k| \le\ssum|a_k| + \ssum|b_k| \mbox{, то } \ssumn|a_k+b_k|$$ --- монотонная ограниченная последовательность и, следовательно, сходится.
\end{proof}
\item
Если ряд $\ssum a_k$ сходится (абсолютно сходится) и $\s$ его сумма, $n_k$ --- строго возрастающая последовательность натуральных чисел, то ряд $\ssum \left(\sum_{j=n_{k-1}+1}^{n_k} a_j\right)$, где $n_0=0$, сходится (сходится абсолютно) и $\s$ --- его сумма.
\begin{proof}
Последовательность частичных сумм сгруппированного ряда --- это подпоследовательность $ S_{n_k} $ последовательности частичных сумм начального ряда.
Последовательность:
$$ \sum_{k=1}^N \left| \sum_{j=n_{k-1}}^{n_N} a_j \right| \le \sum_{j=1}^{n_N}|a_j| $$
ограничена, тогда первая сумма --- монотонная ограниченная последовательность, которая сходится.
\end{proof}
\item
Если члены ряда $a_k \xrightarrow{k \to \infty} 0$, $n_k$ --- строго возрастающая последовательность натуральных чисел и $\sup_k (n_k - n_{k-1}) <\infty)$, 
сгруппированный ряд $\ssum \Big(\sum\limits_{j=n_{k-1}+1}^{n_k} a_j\Big)$, где
$ n_0 = 0 $, сходится $\s$ --- его, сумма то начальный ряд также сходится и $\s$ --- его сумма.
\begin{proof}
Для любого $ m \in \nat$ найдем такое натуральное $r$, что $n_{r-1} < m \le n_r$. Тогда: 
$$ \left| \s_{n_r} - \s_m \right| \le \sum_{k=m+1}^{n_r} |a_k| \le \sum_{k=n_{r-1}+1}^{n_r} |a_k| \le \sum_{k=n_r -l}^{n_r} |a_k|, $$
где $sup_k(n_k-n_{k-1}) \le l$.
Последняя сумма --- $\stackrel{=}o\!(1)$.
Следовательно, если $\s_{n_r}\xrightarrow{r\to\infty}\s$, то $\s_{m} \xrightarrow{m\to\infty}\s$
\end{proof}
\end{enumerate}
